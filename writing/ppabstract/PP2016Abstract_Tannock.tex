% ------------------------------------------------------
% Abstract for talk at Permutation Patterns 2016, Washington D.C. (Based on 2015
% abstract template)
% ------------------------------------------------------
\documentclass[11pt,a4paper]{article}

% 'European' formatting
\frenchspacing \setlength{\parskip}{9pt plus 3pt minus 1pt}
\setlength{\parindent}{0pt}

\setlength{\textwidth}{6.27in}
\setlength{\textheight}{9.69in}
\setlength{\topmargin}{0pt}
\setlength{\headsep}{0pt}
\setlength{\headheight}{0pt}
\setlength{\oddsidemargin}{0pt}
\setlength{\evensidemargin}{0pt}

\usepackage{amsmath}
\usepackage{amssymb}
\usepackage{amsthm}
\usepackage{tikz}

% ------- extra packages
\usepackage{xinttools}
\usepackage{tabularx}
\usepackage{hyperref}
% -------

\newtheorem{theorem}{Theorem}
\newtheorem{corollary}[theorem]{Corollary}
\newtheorem{lemma}[theorem]{Lemma}
\newtheorem{proposition}[theorem]{Proposition}
\newtheorem{conjecture}[theorem]{Conjecture}
\newtheorem{question}[theorem]{Question}
\theoremstyle{definition}
\newtheorem{definition}[theorem]{Definition}
\newtheorem{example}[theorem]{Example}

\newcommand{\talkTitle}[1]{\hrule\section*{\textsf{\textmd{\LARGE #1}}}}
\newcommand{\talkPresenter}[1]{\textbf{\large #1}}
\newcommand{\talkAffiliation}[1]{(#1)}
\newcommand{\talkJointWork}[1]{This talk is based on joint work with #1.}

\renewcommand{\refname}{\normalsize References}
\newenvironment{talkBibliog}{\begin{thebibliography}{9}\small}{\end{thebibliography}}

% ----------------PERSONAL PREAMBLE---------------------
\DeclareMathOperator{\Av}{Av}
\newcolumntype{Y}{>{\centering\arraybackslash}X}
% ------------------------------------------------------
\begin{document}
\begingroup
% ------------------------------------------------------
% Please do not modify above this line.
% Personal macros can be placed here.
% -----------------------------------------------------------------------------
\newcommand{\ie}{\textit{i.e.}}

\newcommand{\av}[2][]{\Av_{#1}\!\left(#2\right)}
\newcommand{\setsize}[1]{%
    \left|#1\right|}

\newcommand{\addp}[3][\emptyset]{\mathtt{add\_point}\left(#2, #3, #1\right)}
\newcommand{\addpe}[1][\emptyset]{\mathtt{add\_point}\left(\varepsilon, #1\right)}

\newcommand{\adda}[2]{\mathtt{add\_ascent}\left(#1, #2\right)}
\newcommand{\addae}[0]{\mathtt{add\_ascent}\left(\varepsilon\right)}
\newcommand{\addd}[2]{\mathtt{add\_descent}\left(#1, #2\right)}
\newcommand{\addde}[0]{\mathtt{add\_descent}\left(\varepsilon\right)}


\newcommand{\perm}[1]{%
    \xintFor ##1 in {#1}\do {##1}
}
\newcommand{\mperm}[2]{%
\left(
\perm{#1},#2\right)
}


% --------------TIKZ MACROS---------------------------
% For shading boxes
\usetikzlibrary{patterns}
\usetikzlibrary{decorations.markings}

\tikzset{->-/.style={decoration={
  markings,
  mark=at position #1 with {\arrow{>}}},postaction={decorate}}}

\providecommand{\picscale}{0.35}
\providecommand{\textscale}{0.175}

\newcommand{\drawthegrid}[1]{%
\draw (0.01,0.01) grid (#1+0.99,#1+0.99);
}

\newcommand{\drawthepoints}[2]{%
\foreach[count=\x] \y in {#1}
\filldraw (\x,\y) circle (#2 pt);
}

\newcommand{\shadeboxes}[1]{
    \foreach \x/\y in {#1}
    \fill[pattern=north east lines, pattern color=black!75] (\x,\y) rectangle +(1,1);

}

\newcommand{\clpattern}[3][5]{%
    \pgfmathsetmacro\circlesize{#1+4}
    \drawthegrid{\xintNthElt{0}{\xintCSVtoList {#3}}}
    \drawthepoints{#3}{#1}
    \foreach \x in {#2}
    \draw (\x,\xintNthElt{\x}{\xintCSVtoList {#3}}) circle (\circlesize pt);
}

\newcommand{\pattern}[4][]{%
    \raisebox{0.6ex}{
    \begin{tikzpicture}[scale=\picscale, baseline=(current bounding box.center), #1]
        \shadeboxes{#4}
        \clpattern[5]{#2}{#3}
    \end{tikzpicture}}\;
}

\newcommand{\textpattern}[4][]{%
    \raisebox{0.6ex}{
    \begin{tikzpicture}[scale=\textscale, baseline=(current bounding box.center), #1]
        \shadeboxes{#4}
        \clpattern[8]{#2}{#3}
    \end{tikzpicture}}\;
}

\newcommand{\modpattern}[4][5]{%
    \shadeboxes{#4}
    \clpattern[#1]{#2}{#3}}

    \newcommand{\decompleft}[5]{%
        \begin{tikzpicture}[scale=1.0, baseline=(current bounding box.center)]
            \fill[pattern=north east lines, pattern color=black!75] (0,-1) rectangle +(-0.5,2);
            \foreach \x/\y in {#1}
                \fill[pattern=north east lines, pattern color=black!75] (\x,\y-1) rectangle +(1,1);
            \draw (0,1) -- (0,-1)
                  (-0.5,0) -- (2,0)
                  (1,1) -- (1,-1);
            \filldraw (0,0) circle (3pt);
            \draw (1.5, 0.5) node {\(#5\)};
            \draw (1.5, -0.5) node {\(#3\)};
            \draw (0.5, 0.5) node {\(#4\)};
            \draw (0.5, -0.5) node {\(#2\)};
        \end{tikzpicture}
    }


% ------------------------------------------------------
% Talk title and presenter info goes here.

\talkTitle{Equivalence classes of mesh patterns with a dominating pattern}
\talkPresenter{Murray Tannock}
\talkAffiliation{Reykjavik University}

% ------------------------------------------------------
% The abstract goes here; please restrict to no more than 2 pages.
% Graphics can be included using TikZ (only).

We establish sufficient conditions for coincidence between mesh patterns when
considered inside a set of avoiders of a classical permutation pattern, and
establish general rules that can be used to construct such coincidences. These
rules along with two special cases allow us to completely classify coincidence
between all sets of avoiders of a mesh pattern of length \(2\) and a classical
pattern of length \(3\). We then go on to consider Wilf-equivalence amongst
mesh patterns of length \(1\) and \(2\) whilst also avoiding the classical pattern
\(231\).

When we consider avoidance of a pattern, \(p\), inside a permutation class we gain
additional restrictions on the occurrences of \(p\).
We will restrict the set of permutations that form the basis
for our permutation class to a single classical pattern, \(\pi\), which we shall
refer to as the \emph{dominating pattern}. In particular, we consider the case
where \(p\) a is mesh pattern of length \(2\) and \(\pi\) is either of the classical
permutation patterns \(231\) or \(321\).

By use of previous results of Claesson, Tenner and Ulfarsson~\cite{ABH}
we can begin to reduce the number of potential coincidences by only looking for
coincidences between patterns that are not coincident before considering the
setting of having a dominating pattern. This gives us an initial upper bound of
\(220\) coincidence classes in each of the four cases we consider, in Table~\ref{tab:domclasses}.

By examining experimental coincidences of non-classical permutation classes,
considered up to length \(11\) we can find rules that allow us to establish
coincidences, and thus establish the number of coincidence classes as follows.

\begin{table}[htb]
\begin{center}
\begin{tabularx}{\textwidth}{c|Y|Y|Y|Y|}
\cline{2-5}
& \multicolumn{4}{|c|}{Dominating Pattern}\\
\cline{2-5}
& \multicolumn{2}{|c|}{\(\av{\perm{2,3,1}}\)}&\multicolumn{2}{|c|}{\(\av{\perm{3,2,1}}\)}\\
\hline
\multicolumn{1}{|r|}{\emph{Underlying Classical pattern}}& \(\perm{1,2}\)&\(\perm{2,1}\)&\(\perm{1,2}\)&\(\perm{2,1}\)\\
\hline
\hline
\multicolumn{1}{|r|}{No Dominating rule}&\(220\)&\(220\)&\(220\)&\(220\)\\
\hline
\multicolumn{1}{|r|}{First Dominating rule}&\(85\)&\(43\)&\(220\)&\(29\)\\
\hline
\multicolumn{1}{|r|}{Second Dominating rule}&\(59\)&\(39\)&\(220\)&\(29\)\\
\hline
\multicolumn{1}{|r|}{Third Dominating rule}&\(56\)&\(39\)&\(220\)&\(29\)\\
\hline
\multicolumn{1}{|r|}{Experimental class size}&\(56\)&\(39\)&\(213\)&\(29\)\\
\hline
\end{tabularx}
\end{center}
    \caption{Coincidence class number reduction by application of Dominating Rules}
    \label{tab:domclasses}
\end{table}

By considering the classical permutation classes \(\av{231}\) and \(\av{321}\)
along with a mesh pattern with underlying classical pattern \(21\), we gain the
following two rules that completely classify the coincidence in theses cases.

\begin{proposition}[First Dominating Pattern Rule]
    \label{prop:dom1}
    Given two mesh patterns \(m_1 =(\sigma, R_1)\) and \(m_2 = (\sigma, R_2)\),
    and a dominating classical pattern \(\pi = (\pi,\emptyset)\) such that
    \(\setsize{\pi} \le \setsize{\sigma} + 1\), the sets \(\av{\{\pi,m_1\}}\) and
    \(\av{\{\pi,m_2\}}\) are coincident if

    \begin{enumerate}
        \item \(R_1 \triangle R_2 = \{(a,b)\}\)
        \item \(\pi \) is contained in \( \addp{\sigma}{(a,b)}\)\label{prop:dom1:cont}
    \end{enumerate}
\end{proposition}

\begin{figure}[hbt]
    \begin{center}
    \begin{tikzpicture}[scale=\picscale]
        \modpattern{}{2,1}{1/0,1/1,1/2,2/2}
    \end{tikzpicture}
        \raisebox{2ex}{\(\mapsto\)}
    \begin{tikzpicture}[scale=\picscale]
        \filldraw (2.5,0.5) circle (3pt);
        \modpattern{}{2,1}{1/0,1/1,1/2,2/2,2/0}
    \end{tikzpicture}

        \caption{Visual depiction of first dominating pattern rule.}
        \label{fig:rule1}
    \end{center}
\end{figure}

\begin{proposition}[Second Dominating Pattern Rule]
    \label{prop:dom2}
    Given two mesh patterns \(m_1 =(\sigma, R_1)\) and \(m_2 = (\sigma, R_2)\),
    and a dominating classical pattern \(\pi = (\pi,\emptyset)\) such that
    \(\setsize{\pi} \le \setsize{\sigma} + 2\), the sets \(\av{\{\pi,m_1\}}\) and
    \(\av{\{\pi,m_2\}}\) are coincident if

    \begin{enumerate}
        \item \(R_1 \triangle R_2 = \{(a,b)\}\)
        \item   \begin{enumerate}
                \item\label{prop:dom2:condc} \(\pi \) is contained in
                \( \adda{\sigma}{(a,b)}\) and some additional conditions on the
                mesh are satisfied that allow selection of different points in
                the occurrence
                \item \(\pi \) is contained in \( \addd{\sigma}{(a,b)}\) and
                some additional conditions on the
                mesh are satisfied that allow selection of different points in
                the occurrence
                        % \begin{enumerate}
                        %     \item \((a+1,b) \in \sigma\) and \((a+1,b-1)\notin R\) and \\
                        %         \((x,b-1)\in R \implies (x,b) \in R \) (where \(x\neq a,a+1\)) and\\
                        %           \((a+1,y)\in R \implies (a,y) \in R\) (where \(y\neq b-1,b\)).
                        %     \item \((a,b+1) \in \sigma\) and \((a-1,b+1)\notin R\) and \\
                        %           \((x,b+1)\in R \implies (x,b) \in R\) (where \(x\neq a-1,a\)) and\\
                        %           \((a-1,y)\in R \implies (a,y) \in R\) (where \(y\neq b,b+1\)).
                        % \end{enumerate}
                    % \item \(\pi \) is contained in \( \addd{\sigma}{(a,b)}\) and
                    %     \begin{enumerate}
                    %         \item \((a+1,b+1) \in \sigma\) and \((a+1,b+1)\notin R\) and \\
                    %               \((x,b+1)\in R \implies (x,b) \in R\) (where \(x\neq a,a+1\)) and\\
                    %               \((a+1,y)\in R \implies (a,y) \in R\) (where \(y\neq b,b+1\)).
                    %         \item \((a,b) \in \sigma\) and \((a-1,b-1)\notin R\) and \\
                    %               \((x,b+1)\in R \implies (x,b) \in R\) (where \(x\neq a-1,a\)) and\\
                    %               \((a-1,y)\in R \implies (a,y) \in R\)  (where \(y\neq b-1,b\)).
                    %     \end{enumerate}
                \end{enumerate}
    \end{enumerate}
\end{proposition}

\begin{figure}[ht]
\centering
\begin{tikzpicture}[scale = 0.75]

     % Make the grid
     \draw[step=1cm,gray!50] (2/3,2/3) grid (5+1/3,5+1/3);

     % Make black cross in the middle
     \draw[-,very thick] (3,2/3) -- (3,5+1/3);
     \draw[-,very thick] (2/3,3) -- (5+1/3,3);
     \draw[-,very thick,gray] (4,2) -- (2,2) -- (2,4) -- (4,4) -- (4,2);

     \fill[black] (3,3) circle (0.1);
     \fill[red] (2,3) circle (0.1);

     % Draw the dots on the side
     \foreach \x in {1,2,4,5}{
         \draw[dotted, gray!50] (\x, 0) -- (\x, 2/3);
         \draw[dotted, gray!50] (\x, 5+1/3) -- (\x, 6);
         \draw[dotted, gray!50] (0,\x) -- (2/3,\x);
         \draw[dotted, gray!50] (5+1/3,\x) -- (6,\x);
        }
     % Draw black dots next to the cross
     \draw[dotted, black, very thick] (3, 0) -- (3, 2/3);
     \draw[dotted, black, very thick] (3, 5+1/3) -- (3, 6);
     \draw[dotted, black, very thick] (0, 3) -- (2/3, 3);
     \draw[dotted, black, very thick] (5+1/3, 3) -- (6, 3);

     \draw[->-=.5,thick] (3,3) to[bend right=15](2,4);

     \draw[->, thick] (5,3) -- (5,4);
     \draw[->, thick] (1,3) -- (1,4);
     \draw[->, thick] (3,5) -- (2,5);
     \draw[->, thick] (3,1) -- (2,1);

     \node [black,right] at (6,3) {$b$};
     \node [black,below] at (2,0) {$a$};

\end{tikzpicture}
    \caption{If the conditions of Proposition~\ref{prop:dom2} are satisfied the box
    \((a,b)\) can be shaded.}
\end{figure}

We then define the notion of containment of a mesh pattern, \(p\), within
another mesh pattern, \(q\), we call this notion \emph{subpattern} containment.
This is defined as an extension of regular mesh pattern containment, with the
additional condition that if \((i,j)\in R\text{ then } R_{ij}\) is contained in
the mesh set of \(q\). This notion is then used to establish the following rule
that fully classifies the coincidences between non-classical permutation classes
where we avoid the classical pattern \(\perm{2,3,1}\) and a mesh pattern with
underlying classical pattern \(\perm{1,2}\).
\begin{proposition}[Third Dominating Pattern Rule]
    \label{prop:dom3}
    Given two mesh patterns \(m_1 =(\sigma, R_1)\) and \(m_2 = (\sigma, R_2)\),
    and a dominating classical pattern \(\pi = (\pi,\emptyset)\), the sets
    \(\av{\{\pi,m_1\}}\) and \(\av{\{\pi,m_2\}}\) are coincident if
    \begin{enumerate}
        \item \(R_1 \triangle R_2 = \{(a,b)\}\)
        \item\label{prop:dom3:condocc} \(\addp[D]{\mperm{\sigma}{R_1}}{(a,b)}\)
            where \(D\in\{N,E,S,W\}\)
            is coincident with a mesh pattern containing an occurrence of
            \(\mperm{\sigma}{R_2}\) as a subpattern.
    \end{enumerate}
\end{proposition}

These three rules, along with two special cases, give a complete coincidence
classification of the non-classical permutation classes \(\av{\pi,p}\) where
\(\pi\) is a fixed classical permutation pattern of length \(3\) and \(p\) is a
mesh pattern of length \(2\).

We then go on to consider Wilf-equivalences between non-classical permutation
classes where we avoid the permutation pattern \(\perm{2,3,1}\) and
a mesh pattern of length \(2\). In order to establish these Wilf-equivalences
we use the results of the prior section as well as some observations about
symmetries in order to reduce the initial upper bound on Wilf-equivalence classes.
Using the bijection between avoiders of the vincular pattern
\textpattern{}{2,3,1}{1/0,1/1,1/2,1/3} and set partitions presented by
Claesson~\cite{A}, and structural decomposition of avoiders to obtain generating
functions we fully classify the Wilf-equivalences. This results in \(13\)
non-trivial (\ie{} not containing only a single coincidence class) Wilf-equivalence
classes. When using generating function methods we also provide enumeration of
the non-classical permutation classes.

\begin{example}
  The following patterns are experimentally Wilf-equivalent up to length 10 in
\(\av{\perm{2,3,1}}\)
\begin{equation*}
    m_1 = \pattern{}{1,2}{1/0,1/1,1/2,2/0,2/1} \text{ and }
    m_2 = \pattern{}{2,1}{0/1,1/0,1/1,1/2,2/1}
\end{equation*}
    Considering structural decomposition of an avoider of \(m_1\) in \(\av{\perm{2,3,1}}\)
around the leftmost point, we find that the set of avoiders has form
\begin{equation*}
    \mathcal{M}_1 = \varepsilon \sqcup
    \decompleft{0/1,1/0}{\mathcal{M}_1}{}{}{\mathcal{M}^\prime_1}
\end{equation*}
Where \(\mathcal{M}^\prime_1\) is a permutation avoiding \(\perm{2,3,1}, m_1\)
and \(\textpattern{}{1}{0/0,0/1,1/0}\), these permutations have form
\(\mathcal{M}^\prime_1\)
\begin{equation*}
    \mathcal{M}^\prime_1 = \varepsilon \sqcup
    \decompleft{0/1,1/0}{\scriptstyle \mathcal{M}_1\setminus\varepsilon}{}{}{\mathcal{M}^\prime_1}
\end{equation*}
    The avoiders of \(m_2\) in \(\av{\perm{2,3,1}}\) can be split in a similar manner
around the leftmost point. This gives us that both of these sets have generating
function \(M(x)\) satisfying
\begin{align}
    M(x) &= 1 + xM(x)M^\prime(x) \label{eqn:Pgen}\\
    M^\prime(x) &= 1 + x(M(x)-1)M^\prime(x)\label{eqn:PprimeGen}
 \end{align}

 Solving \ref{eqn:PprimeGen} for \(M^\prime(x)\) and substituting into
 \ref{eqn:Pgen} gives us that the generating function for
 \(M(x)\) satisfies
 \begin{equation}
     M(x) = xM^2(x) - x(M(x) - 1) + 1
 \end{equation}
 Solving \(M(x)\) and evaluating the coefficients gives the sequence
 \begin{equation*}
    1, 1, 1, 2, 4, 9, 21, 51, 127, 323, 835,\dotsc \tag{\href{https://oeis.org/A001006}{OEIS: A001006 with offset 1}}
\end{equation*}
This is an offset of the Motzkin numbers.
\end{example}


% ------------------------------------------------------
% Please comment out if not joint work.

\talkJointWork{Henning Ulfarsson}

% ------------------------------------------------------
% Please comment out if no references.

\begin{talkBibliog}
 % \bibitem{comtet} L.~Comtet, Advanced combinatorics, D. Reidel Publishing Co., Dordrecht (1974).
 \bibitem{ABH}
  A.~Claesson, B.~E.~Tenner, and H.~Ulfarsson,
  Coincidence among families of mesh patterns,
  \emph{CoRR}, 2014.
  \url{http://arxiv.org/abs/1412.0703}
 \bibitem{A}
 A.~Claesson
 \newblock Generalized pattern avoidance
 \emph{Eur. J. Comb}, 2001.
 \url{http://dx.doi. org/10.1006/eujc.2001.0515}
 % \bibitem{flanoy} P.~Flajolet and M.~Noy, Analytic combinatorics of non-crossing configurations, Discrete Math., 204 (1999), 203--229
\end{talkBibliog}

% ------------------------------------------------------
\endgroup
\end{document}
% ------------------------------------------------------
