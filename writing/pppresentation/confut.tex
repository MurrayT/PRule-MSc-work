\section{Conclusions and Future Work}
\label{sec:Conclusions and Future Work}
\subsection{Conclusions}
\label{sub:Conclusions}
\begin{frame}{Conclusions}
  \begin{itemize}
  \item Automatic coincidence classification is difficult.
  \begin{itemize}
    \item Completely classified for length 2 mesh patterns and length 3 classical patterns.
    \item Gets harder with larger dominating patterns.
  \end{itemize}
  \item There are a number of Wilf-classes that give interesting
  enumerations.
  \begin{itemize}
    \item Would be interesting to try and connect these sets to different objects.
  \end{itemize}
  \end{itemize}
\end{frame}
\subsection{Future Work}
\label{sub:Future Work}
\subsubsection{Extensions of rules}
\label{subs:Extensions of rules}
\begin{frame}{Future Work: Extensions of rules}
    One can consider application of the Third Dominating rule, as well as a
    simple extension, without any dominating pattern. This can capture the special
    case described by Claesson, Tenner, and Ulfarsson in \cite{ABH}, that was
    unexplained by the general results proved in that paper.
    \begin{equation*}
     m_1 = \pattern{}{1,2}{0/1,0/2,1/1,1/2,2/0}, \text{ and } m_2 = \pattern{}{1,2}{0/2,1/0,1/1,2/0,2/1}
   \end{equation*}

   We cannot apply the first, or second, rule without a dominating pattern.

   We can also consider taking sets of mesh patterns, or sets of dominating patterns.
\end{frame}

\subsubsection{Equivalences with different dominating patterns}
\label{subs:Equivalences with different dominating patterns}
\begin{frame}{Future Work: Equivalences with different dominating patterns}
  It would be interesting to consider Wilf-equivalences
amongst classes where \(\perm{3,2,1}\) is the dominating pattern.

It is also interesting to consider Wilf-equivalence when we have different dominating
patterns.
We can show that the sets \(\mathcal{T} = \av{\textpattern{}{1,2}{1/2,2/0,0/0,1/0,1/1,2/1},\perm{2,3,1}}\) and
\(\mathcal{U} = \av{\textpattern{}{1,2}{1/2,2/0,0/0,1/0,1/1,2/1},\perm{3,2,1}}\)
are Wilf-equivalent.
\pause
\begin{equation*}
    \raisebox{0.6ex}{
    \begin{tikzpicture}[scale=1, baseline=(current bounding box.center)]
        \draw (0.5,1.5) node {\(b_1\)};
        \draw (0.5,2.5) node {\( b_2\)};
        \draw (2.5,2.5) node {\( b_3\)};
        \modpattern[2]{2}{1,2}{0/0,1/0,1/1,1/2,2/0,2/1}
    \end{tikzpicture}
    }
\end{equation*}
\end{frame}

\subsubsection{Other questions}
\label{subs:Other questions}
\begin{frame}{Future Work: Other question}
\begin{itemize}
  \item When can a mesh pattern be contained precisely once in a permutation?
  \item What nice properties does the subpattern relation have?
\end{itemize}
\end{frame}
