\section{Coincidences amongst families of mesh patterns and classical patterns}
One interesting question to ask about permutation patterns considers when a
pattern may be avoided by, or contained in, arbitrary permutations. Two
patterns \(\pi\) and \(\sigma\) are said to be \emph{coincident} if the set
of permutations that avoid \(\pi\) is the same as the set of permutations that
avoid \(\sigma\). This extends to sets of patterns as well as single patterns.

We consider the avoidance sets, \(\av{p,q}\) where \(p\) is a classical pattern
of length \(3\) and \(q\) is a mesh pattern of length \(2\) in order to
establish some rules about when these two sets give the same avoidance set.

We first define some operations on mesh patterns.
\begin{definition}
Given a pattern \(p\), let \(\addp[D]{p}{(a,b)}\) be the operation that returns
a mesh pattern equivalent to placing a point in the center of box \((a,b)\) in
\(p\), with shading defined by \(D\subseteq\{N,E,S,W\}\).
\end{definition}
The set \(D\) defines the shading by indicating that the boxes in the cardinal
directions in \(D\) next to the point are shaded in the resulting pattern.
Since there is no ambiguity we let \(\addpe[D]\) be equivalent to \(\addp[D]{\varepsilon}{(0,0)}\).
This operation fails if the box \((a,b)\) is in the mesh set of \(p\).

\begin{example}
    The result of adding a single point to the empty permutation for each cardinal direction.
 \begin{align*}
     \addpe[\{N\}] = \textpattern{}{1}{0/1,1/1} &\qquad\addpe[\{E\}] =\textpattern{}{1}{1/1,1/0} \qquad \\
     \addpe[\{S\}] = \textpattern{}{1}{0/0,1/0} &\qquad\addpe[\{W\}] =\textpattern{}{1}{0/0,0/1} \qquad
\end{align*}
    A more complex example for \texttt{add\_point}
\begin{equation*}
\addp[\{N,E\}]{\pattern{}{2,1,3}{0/3}}{(2,3)} = \pattern{}{2,1,4,3}{0/3,0/4,2/4,3/4,3/3}
\end{equation*}
\end{example}

\begin{definition}
    Given a pattern \(p\), define \(\addd{p}{(a,b)}\), and \(\addi{p}{(a,b)}\),
    as the operations that return a mesh pattern equivalent to placing an
    decrease, or increase, in the center of box \((a,b)\) in \(p\).
\end{definition}

\begin{example}
    \begin{align*}
        \addie = \pattern{}{1,2}{0/1,1/0,1/1,1/2,2/1}\\
        \addde = \pattern{}{2,1}{0/1,1/0,1/1,1/2,2/1}
    \end{align*}
\end{example}

We now attempt to fully classify coincidences in families characterised by avoidance
of a classical pattern of length \(3\) and a mesh pattern of length \(2\).

It can be easily seen that in order to classify set equivalences one need only
consider coincidences within the family of mesh patterns with the same underlying
classical pattern, this is due to the fact that \(\perm{2,1}\in\av{\mperm{1,2}{R}}\)
and \(\perm{1,2}\in\av{\mperm{2,1}{R}}\) for all mesh-sets \(R\). So
\(\av{\mperm{1,2}{R}}\setminus \av{\mperm{2,1}{S}}\neq \emptyset\ \forall
R,S\in [0,2]\times [0,2]\), and hence the sets are disjoint.

We know that there are a total of \(512\) mesh-sets for each underlying classical
pattern. By use of the shading lemma, simultaneous shading lemma, and
one special case, we can reduce the number of equivalence classes to \(220\).

\subsection{Equivalence classes of Av(\{321, (21, \textit{R})\}).}

\begin{proposition}[First Dominating Pattern Rule]
    \label{prop:dom1}
    Given two mesh patterns \(m_1 =(\sigma, R_1)\) and \(m_2 = (\sigma, R_2)\),
    and a dominating classical pattern \(\pi = (\pi,\emptyset)\) such that
    \(\setsize{\pi} \le \setsize{\sigma} + 1\), the sets \(\av{\{\pi,m_1\}}\) and
    \(\av{\{\pi,m_2\}}\) are coincident if

    \begin{enumerate}
        \item \(R_1 \triangle R_2 = \{(a,b)\}\)
        \item \(\pi \preceq \addp{\sigma}{(a,b)}\)\label{prop:dom1:cont}
    \end{enumerate}
\end{proposition}
In order to prove this proposition we must first make the following note.

\begin{note}
    \label{not:downcmesh}
    Let \(R^\prime \subseteq R\). Then any occurence of \((\tau, R)\) in a permutation
    is an occurence of \((\tau,R^\prime)\).
\end{note}
\begin{proof}
    Assume without meaningful loss of generality that \(R^\prime\) is a proper
    subset of \(R\).

    Consider an occurence of \((\tau, R)\) in a permutation \(\sigma\), obviously
    this corresponds to an occurrence of \(\tau\) in \(\sigma\). Now consider
    the mesh sets \(R\) and \(R^\prime\), since \(R^\prime \subseteq R\) then
    there are more restrictions on where points are in an occurrence of \((\tau,R)\).
    Namely, for every shaded box in \(R\) the corresponding region in \(\sigma\)
    must contain no points, since \(R^\prime\) has less shading than \(R\) there
    exists a region in the occurence of \((\tau,R)\) in \(\sigma\) that is
    now devoid of restrictions. However, by removing restrictions we cannot
    make an occurence become not an occurence, and therefore the same occurrence
    of \(\tau \) in \(\sigma\) is now an occurence of \((\tau, R^\prime)\).

\end{proof}
\begin{proof}[Proof of Proposition~\ref{prop:dom1}]

    We need to prove that \(\av{\{\pi,m_1\}} = \av{\{\pi,m_2\}}\), or equivalently
    \(\mathfrak{S}\setminus \av{\{\pi,m_1\}} = \mathfrak{S}\setminus \av{\{\pi,m_1\}}\).

    Assume without meaningful loss of generality that \(R_2 = R_1 \cup \{(a,b)\}\).

    Consider a permutation \(\omega\) that contains an occurrence of \(m_2\)
    by Note~\ref{not:downcmesh} any of these occurrences is also an occurence
    of \(m_1\). This proves that every occurence of \(m_2\) is also an
    occurence of \(m_1\) and \(\mathfrak{S}\setminus \av{\{\pi,m_2\}}
    \subseteq \mathfrak{S}\setminus \av{\{\pi,m_1\}}\).

    Now we consider a permutation \(\omega^\prime\) containing an occurence of
    \(m_1\). Consider placing a point in the region corresponding to the box
    \((a,b)\), regardless of where in this region we place the point by
    condition~\ref{prop:dom1:cont} of the Proposition we create an occurrence
    of \(\pi\), therefore there can be no points in this region, which
    could have been represented in the mesh set \(R_1\) by adding the box
    \((a,b)\), and therefore every occurrence of \(m_1\) is in fact an
    occurrence of \(m_2\).\qedhere


\end{proof}

\subsection{Equivalence classes of Av(\{231, (21, \textit{R})\}).}

\begin{lemma}
    Given a mesh pattern \(m =(\sigma, R)\), where the box \((a,b)\) is not
    in \(R\), and a dominating classical pattern \(\pi = (\pi,\emptyset)\) if
    \(\pi \preceq \addi{\sigma}{(a,b)}\)(\(\pi \preceq \addd{\sigma}{(a,b)}\))
    then in any occurrence of \(m\) in a permutation \(\varrho\) the region
    corresponding to the box \((a,b)\) can only contain an increasing
    (decreasing) subsequence of \(\varrho\).
\end{lemma}
The proof is analogous to the proof of Proposition~\ref{prop:dom1}.

\begin{proposition}[Second Dominating Pattern Rule]
    \label{prop:dom2}
    Given two mesh patterns \(m_1 =(\sigma, R_1)\) and \(m_2 = (\sigma, R_2)\),
    and a dominating classical pattern \(\pi = (\pi,\emptyset)\) such that
    \(\setsize{\pi} \le \setsize{\sigma} + 2\), the sets \(\av{\{\pi,m_1\}}\) and
    \(\av{\{\pi,m_2\}}\) are coincident if

    \begin{enumerate}
        \item \(R_1 \triangle R_2 = \{(a,b)\}\)
        \item   \begin{enumerate}
                    \item \(\pi \preceq \addi{\sigma}{(a,b)}\) and
                        \begin{enumerate}
                            \item \((a+1,b) \in \sigma\) and \\
                                \((x,b-1)\in R \implies (x,b) \in R \) (where \(x\neq a,a+1\)) and\\
                                  \((a+1,y)\in R \implies (a,y) \in R\) (where \(y\neq b-1,b\)).
                            \item \((a,b+1) \in \sigma\) and \\
                                  \((x,b+1)\in R \implies (x,b) \in R\) (where \(x\neq a-1,a\)) and\\
                                  \((a-1,y)\in R \implies (a,y) \in R\) (where \(y\neq b,b+1\)).
                        \end{enumerate}
                    \item \(\pi \preceq \addd{\sigma}{(a,b)}\) and
                        \begin{enumerate}
                            \item \((a+1,b+1) \in \sigma\) and \\
                                  \((x,b+1)\in R \implies (x,b) \in R\) (where \(x\neq a,a+1\)) and\\
                                  \((a+1,y)\in R \implies (a,y) \in R\) (where \(y\neq b,b+1\)).
                            \item \((a,b) \in \sigma\) and \\
                                  \((x,b+1)\in R \implies (x,b) \in R\) (where \(x\neq a-1,a\)) and\\
                                  \((a-1,y)\in R \implies (a,y) \in R\)  (where \(y\neq b-1,b\)).
                        \end{enumerate}
                \end{enumerate}
    \end{enumerate}
\end{proposition}
\begin{proof}
\end{proof}
