If we consider a similar approach to dominating patterns of length \(4\) and
mesh patterns of length \(2\), it can be seen that the number of cases required
to establish rules increases to a number that is infeasible to compute manually.
For an extension of the First Dominating rule alone, we would have to consider
placement of points in any pair of unshaded regions. The fact that the rules
established do not completely cover the coincidences with a dominating pattern
of length \(3\) shows that this is a difficult task.

It is interesting to consider the application of the Third Dominating rule, as
well as the simple extension of allowing a sequence of point insertions, to mesh
patterns without any dominating pattern in order to try to capture some of the
coincidences described in
\textcite{DBLP:journals/combinatorics/HilmarssonJSVU15} and
\textcite{DBLP:journals/corr/ClaessonTU14}.
 \begin{example}
   The coincidence of the patterns
   \begin{equation*}
     m_1 = \pattern{}{1,2}{0/1,0/2,1/1,1/2,2/0}, \text{ and } m_2 = \pattern{}{1,2}{0/2,1/0,1/1,2/0,2/1}
   \end{equation*}
	does not follow from the general methods presented by
	\textcite{DBLP:journals/corr/ClaessonTU14}, but is rather handled there as a
	special case. We can do it as follows:
   Consider a permutation containing \(m_1\),
   \begin{equation*}
     \begin{tikzpicture}[scale=0.5]
         \modpattern[5]{}{1,2}{0/1,0/2,1/1,1/2,2/0}
         \draw (1.5,0.5) node {\(Y\)};
         \draw (2.5,1.5) node {\(X\)};
     \end{tikzpicture}
   \end{equation*}
   If the regions corresponding to both \(X\) and \(Y\) are empty then we have
   an occurrence of \(m_2\).
   If the region corresponding to \(X\) is non-empty, we can then choose
   the lowest valued point in this region
   \begin{equation*}
     \begin{tikzpicture}[scale=0.5]
         \modpattern[5]{1,3}{1,3,2}{0/1,0/2,0/3,1/1,1/2,1/3,2/0,2/1,3/0,3/1}
         \draw (1.5,0.5) node {\(Y\)};
     \end{tikzpicture}
   \end{equation*}
   If the region corresponding to \(Y\) is empty then we have an occurrence of
   \(m_2\) with the indicated points.
   Now if the region corresponding to \(Y\) is non-empty, we can choose the
   rightmost point in this region.
   \begin{equation*}
     \begin{tikzpicture}[scale=0.5]
         \modpattern[5]{2,4}{2,1,4,3}{0/2,0/3,0/4,1/2,1/3,1/4,2/0,2/1,2/2,2/3,2/4,3/0,3/1,3/2,4/0,4/1,4/2}
     \end{tikzpicture}
   \end{equation*}
   And now the two indicated points form an occurrence of \(m_2\).
   We have therefore shown that any occurrence of \(m_1\) leads to an occurrence of
   \(m_2\) and we can easily show the converse by the same reasoning, so \(m_1\)
   and \(m_2\) are coincident.
   This is captured by an extension of the Third Dominating rule where we allow
   multiple steps of adding points before we check for subpattern containment.
 \end{example}

It would be interesting to consider a systematic explanation of
Wilf-equivalences among classes where \(\perm{3,2,1}\) is the dominating
pattern, possibly using the  construction presented in
\textcite[Sec.~11]{2015arXiv151203226B}, in order to directly reach enumeration
and hopefully establish some of the non-trivial Wilf-equivalences between
classes with different dominating patterns.
