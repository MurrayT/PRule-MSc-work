\chapter[Wilf-Classification]{Wilf-equivalences under dominating  patterns}

Another question often asked in thefield of permutation patterns is that
of Wilf-equivalence. Two patterns \(\pi\) and \(\sigma\) are said to be
Wilf-equivalent if their avoidance sets have the same size at each
length. More formally

\begin{definition}[Wilf-equivalence]
    Two patterns \(\pi\) and \(\sigma\) are said to be \emph{Wilf-equivalent}
    if for all \(k_{} \ge 0\), \(\size{\av[k]{\pi}} = \size{\av[k]{\sigma}}\).

    Two sets of permutation patterns \(R\) and \(S\) are are
    \emph{Wilf-equivalent} if for all \(k_{} \ge 0\),
    \(\size{\av[k]{R}} = \size{\av[k]{S}}\).
\end{definition}

Wilf-equivalence is of interest as if two permutation classes are enumerated
in the same way then there should exist a bijection between them, and therefore
any other combinatorial object that they represent.

Coincident pattern classes are also Wilf-equivalent. This is due to the fact that
if \(\av[k]{S} = \av[k]{R}\) then obviously \(\size{\av[k]{R}} = \size{\av[k]{S}}\).
Coincidence is therefore a stronger equivalence condition than Wilf-equivalence.

There are a number of symmetries we can use when examining Wilf-equivalences
to reduce the amount of work, it can be easily seen that the reverse, complement
and inverse operations preserve enumeration, and therefore these classes are trivially
Wilf-equivalent.

\begin{align*}
    \rev{\textpattern{}{2,3,1}{}} &= \textpattern{}{1,3,2}{}\\
    \com{\textpattern{}{2,3,1}{}} &= \textpattern{}{2,1,3}{}\\
    \inv{\textpattern{}{2,3,1}{}} &= \textpattern{}{3,1,2}{}
\end{align*}

The group of symmetries on permutations is isomorphic to the dihedral group
of order \(8\), the group of symmetries of a square. Composition of the above
symmetries gives the remaining 5 symmetries. If we consider generators of the
group the operations \emph{reverse-inverse} and \emph{reverse} correspond to
the generators of the dihedral group.

Since we are always considering Wilf-equivalences in the set \(\av{S}\) we
must only use these symmetries when they preserve the dominating pattern.

Throughout this section we will consider Wilf-equivalences of patterns
whilst avoiding the \emph{Dominating Pattern} \(\perm{2,3,1}\). We will
use \(\mathcal{C}\) to denote the set of these avoiders and \(C(x)\) will
be the usual Catalan generating function satisfying \(C(x) = 1 + C(x)^2\).
This is easy to see by structural decomposition around the maximum, as
shown in \FigureRef{fig:decompmax}

\begin{figure}[!ht]
    \centering
    \decompmax{0/1,1/0}{A}{}{}{B}
    \caption{Structural decomposition of a typical avoider of 231}
    \label{fig:decompmax}
\end{figure}

The elements to the left of the maximum, \(A\), have the structure of a \(231\)
avoiding permutation, and the elements to the right of the maximum, \(B\), have the
structure of a \(\perm{2,3,1}\) avoiding permutation. Furthermore, all the
elements in \(A\) lie below all of the elements in \(B\).

We can also decompose a permutation avoiding \(\perm{2,3,1}\) around the leftmost
point, giving a similar figure.

\section{Wilf-classes with patterns of length 1.}
When considering the mesh patterns of length \(2\) it will be useful to know
the Wilf-equivalence classes of the mesh patterns of length \(1\) inside
\(\av{\perm{2,3,1}}\).

It can be seen that the patterns in the following set are set equivalent.
\begin{equation*}
    \left\{
        \begin{array}{c}
        \textpattern{}{1}{},
        \textpattern{}{1}{0/0},
        \textpattern{}{1}{1/0},
        \textpattern{}{1}{0/0,1/0},
        \textpattern{}{1}{0/1},\\
        \textpattern{}{1}{0/0,0/1},
        \textpattern{}{1}{1/1},
        \textpattern{}{1}{1/0,1/1},
        \textpattern{}{1}{0/1,1/1}
    \end{array}
    \right\}
\end{equation*}
This is due to the fact that every permutation except the empty
permutation must contain an occurrence of all of these patterns.

The pattern \(\textpattern{}{1}{0/0,1/0,0/1,1/1}\) is in its own
Wilf-class since the only permutation containing this pattern is
the permutation \(\perm{1}\). The avoiders of this pattern therefore
have generating function \(\var(x) = C(x)-x\).\nextvar

The avoiders of the pattern \(p_1 = \textpattern{}{1}{0/1,1/0}\) can be
decomposed around the maximum element in order to give the following
structural decomposition.
\begin{equation*}
    \mathcal{\var} = \varepsilon \sqcup \decompmax{0/1,1/0}{\mathcal{\var}}{}{}{\mathcal{C}\setminus\varepsilon}
\end{equation*}
Since if the upper right section was empty the maximum would create an
occurrence of the pattern, however no points in this section can create
an occurrence since the maximum lies in a region corresponding to the
shading in \(p_1\). The lower right region however can create occurrences
of \(p_1\) and therefore must also avoid \(p_1\) as well as \(\perm{2,3,1}\).
This gives the generating function of avoiders to be the function \(\var(x)\)
satisfying.
\begin{align*}
    \var(x) &= 1 + x\var(x)(C(x)-1) \\
    \shortintertext{Solving for \(\var\) gives}
    \var(x) &= \frac{1}{1+x-xC(x)}
\end{align*}

Evauluation of this generating function gives the Fine numbers
(\href{https://oeis.org/A000957}{OEIS: A000957}). This pattern is one of the
quadrant marked mesh patterns studied by Remmel, Kitaev and TiefenbruckZ\cite{2012arXiv1201.6243K}.

It can be shown by use of \PropositionRef{prop:dom2} that the patterns
\textpattern{}{1}{0/0,1/1} and \(p_2 = \textpattern{}{1}{0/0,0/1,1/1}\) are
coincident. Consider the decomposition of an avoider of \(p_2\) in
\(\av{\perm{2,3,1}}\) around the maximum\nextvar[\varullr]
\begin{equation*}
    \mathcal{\var}_1 =\varepsilon \sqcup \decompmax{0/1,1/0}{\mathcal{C}\setminus\varepsilon}{}{}{\mathcal{C}}
\end{equation*}
This can be explained succinctly by the fact that a permutation containing
\(p_2\) starts with it's maximum, by not allowing the left part of the
\(\perm{2,3,1}\) avoider to be empty we prevent an occurrence from ever happening.

Consider \(p_3 = \textpattern{}{1}{0/1,1/0,1/1}\), avoiding this pattern
means that a permutation does not end with it's maximum. We can perform a similar
decomposition as before to get
\begin{equation*}
    \mathcal{\var}_2 =\varepsilon \sqcup \decompmax{0/1,1/0}{\mathcal{C}}{}{}{\mathcal{C}\setminus\varepsilon}
\end{equation*}

Now consider \(p_4 = \textpattern{}{1}{0/0,0/1,1/0}\), the avoiders of this
pattern are permutations that do not start with their minimum. In this case we
perform the decomposition around the leftmost element
\begin{equation*}
    \mathcal{\var}_3 =\varepsilon \sqcup \decompleft{0/1,1/0}{\mathcal{C}\setminus\varepsilon}{}{}{\mathcal{C}}
\end{equation*}
All of these classes have the same generating function namely
\begin{equation*}
    \var(x) = 1 + xC(x)(C(x)-1)
\end{equation*}

There is one pattern of length \(1\) still to consider. The pattern
\(p_5 = \textpattern{}{1}{0/0,1/0,1/1}\) is avoided by all permutations
that do not end in their minimum. Considering the standard decomposition
of a \(\perm{2,3,1}\) avoider around the maximum we can see that an avoider
of \(p_5\) must fit into the following form precisely once.
\begin{equation*}
    \mathcal{H} = \varepsilon \sqcup
    \underbrace{\decompmax{0/1,1/0}{\mathcal{C}\setminus\varepsilon}{}{}{\mathcal{C}}}_\text{
        Minimum comes before the maximum.
    } \sqcup
    \underbrace{\decompmax{0/0,0/1,1/0}{}{}{}{\mathcal{H}}}_\text{
        Minimum is after the maximum, cannot be last.
    }
\end{equation*}
Therefore this particular class has generating function \(H(x)\) satisfying
\begin{equation*}
    H(x) = 1 + xC(x)(C(x)-1) + xH(x)
\end{equation*}

\section{Wilf-classes with patterns of length 2}
By use of the set equivalences from \ChapterRef{chap:coincs} we know there are
at most \(95\) Wilf-equivalence classes.

In order to consider symmetries we must only take the symmetries that preserve
the pattern \(\perm{2,3,1}\). If we take any of the symmetries alone the
permutation is different. The only symmetry that preserves the pattern
\(\perm{2,3,1}\) is that of \emph{reverse-complement-inverse}. Using this
set of symmetries to merge classes gives us \(61\) classes of trivial
Wilf-equivalences.

Computing avoiders up to length \(10\) gives us \(23\) Wilf-classes,
of which \(13\) are non-trivial.

When considering explanations of Wilf-equivalences we consider how the
permutations correspond to set-partitions.

\begin{note}
    \label{not:setequiv}
    The avoiders of the pattern \(\mperm{2,3,1}{\{(1,0),(1,1),(1,2),(1,3)\}}\)
    in \(\mathfrak{S}_n\) are in one-to-one correspondence with partitions of
    \(\nrange{n}\). \cite{DBLP:journals/ejc/Claesson01}
\end{note}
\begin{proof}
    Let \(\pi\) be a permutation in \(\av[n]{\perm{2,3,1}}\) take the permutation
    in one-line notation and insert a dash between each ascent in \(\pi\). This
    corresponds to set partitions where the blocks are the elements between the
    dashes, the blocks are listed in increasing order of their least element,
    with the elements written in each block in descending order.
\end{proof}

\begin{example}
    Given the permutation \(\pi = \perm{5,4,2,1,3,9,6,8,7}\) this corresponds to the
    partition \(\{\{1,2,4,5\},\{3\},\{6,9\},\{8,7\}\}\).
\end{example}

We are looking at permutations in \(\av{\perm{2,3,1}}\), all of these permutations
also avoid the mesh pattern in \NoteRef{not:setequiv}.

\subsection{}
The set/symmetry classes containing the following patterns are
Wilf-equivalent in \(\av{\perm{2,3,1}}\) but are not set equivalent or
symmetries of each other
\begin{equation*}
    m_1 = \pattern{}{1,2}{0/0,0/2,
                    1/0,1/1,
                    2/0,2/1,2/2}
    \text{ and }
    m_2 = \pattern{}{1,2}{0/0,
                    1/0,1/1,1/2,
                    2/0,2/1,2/2}
\end{equation*}
In this case it is better to consider the containers of the patterns
instead of the avoiders due to the amount of shadings in the mesh.

We look at the containers of the pattern \(m_1\), there can only ever
be one occurrence of this pattern in a permutation corresponding to
the last point in the permutation and the minimum. The only form that
points in either of the two boxes can take is a decreasing sequence.
For a permutation of length \(k\) if we fix the number of points in
one of the boxes the number of points in the other box is determined.
Therefore we can have any number of points from \(\setrange{0}{k-2}\)
points in the bottom box. Therefore there are \(k-1\) containers of
length \(k\). These permutations correspond to set partitions of \(k\)
points into exactly two non-overlapping parts partitioned by the first
element and the minimum.

Now consider the containers of \(m_2\), we know that the unshaded region
must contain a decreasing subsequence, with the point corresponding to
the \(1\) in the mesh pattern. This decreasing subsequence has \(k-1\)
points, we can put the point corresponding to the \(2\) above any of these
points and therefore there are \(k-1\) containers of length \(k\).

Therefore these two patterns have been shown to have the same number of
avoiders of length \(k\) for all \(k\) and therefore all Wilf-equivalent.

\subsection{}
The classes containing the following patterns are Wilf-equivalent
when avoiders are considered in \(\av{\perm{2,3,1}}\)
\begin{align}
    m_1 = \pattern{}{1,2}{0/0,0/1,1/1,1/2,2/0,2/1} &\text{ and }
    m_2 = \pattern{}{2,1}{0/0,0/1,1/1,1/2,2/0,2/1} \label{eq:singleud}\\
    \text{ and }
    m_3 = \pattern{}{2,1}{0/1,0/2,1/0,1/1,1/2,2/1} &\text{ and }
    m_4 = \pattern{}{2,1}{0/1,0/2,1/0,1/1,1/2,2/0} \label{eq:other}
\end{align}

First we consider the coincidence in two parts then consolidate these parts.
Consider the coincidence shown in \eqref{eq:singleud}. The easiest way to
show that these are equinumerous is to consider the containers in the realm
of set partitions.

Due to the shading we know the following about the points corresponding
to the points in the patterns.
\begin{itemize}
    \item The point corresponding to the first point in both patterns
        must lie in the first block of our set partition (there are
        no points southwest from it in the permutation).
    \item The point corresponding to the second point in both patterns
        is a block bottom (there are no points southeast of it in
        the permutation).
    \item The block containing the point corresponding to the second
        point in both patterns contains only the point (it is a singleton
        block).
\end{itemize}
This tells us that an occurence of the patterns must happen when there
is a singleton block occurring after the first block. The difference
between the patterns is in the underlying classical pattern. This
means that permtuations containing \(m_1\) correspond to set partitions
with a singleton block with value one higher than some element in the
block containing \(1\). The permutations containing \(m_2\) correspond
to the set partitions containing a block with block bottom having value
one lower than some element in the block containing \(1\) and if this
block is not the block containing \(1\) then it is a singleton block.

Consider an avoider of \(\perm{2,3,1}\) and \(m_3\). We can perform
the decomposition around the maximum
\begin{equation*}
    \mathcal{I}_1 = \varepsilon \sqcup
    \decompmax{1/0,0/1}{\mathcal{I}_1}{}{}{\mathcal{G}_1}
\end{equation*}
Since only the first point in the top right region can
create an occurrence of \(m_3\) if and only if it is the
maximum in this region we must avoid starting with the maximum.

Looking at avoiders of \(\perm{2,3,1}\) and \(m_4\) we
can perform the same decomposition around the maximum to get
\begin{equation*}
    \mathcal{I}_2 = \varepsilon \sqcup
    \decompmax{1/0,0/1}{\mathcal{I}_2}{}{}{\mathcal{G}_3}
\end{equation*}
Since an occurrence of \(m_4\) can never occur in the
top right region, and could only occur between the first point
in the region and the maximum, if and only if this first point is the minimum.
Since both \(\mathcal{G}_1\) and \(\mathcal{G}_3\) have the same
enumeration, \(\mathcal{I}_1\) and \(\mathcal{I}_2\) must also have
the same enumeration and are therefore Wilf-equivalent.

Now we must consolidate these two subclasses. In order to do this we must
consider the decomposition around the leftmost point of a permutation in
\(\av{\perm{2,3,1},m_1}\) we gain the following.
\begin{equation*}
    \mathcal{I}_3 = \varepsilon \sqcup
    \decompleft{1/0,0/1}{\mathcal{I}_3}{}{}{\mathcal{G}_3}
\end{equation*}
It is therefore obvious that avoiders of \(m_1\) and avoiders of \(m_4\)
have the same enumeration, and therefore all four patterns are Wilf-equivalent
in \(\av{\perm{2,3,1}}\) with generating function satisfying
\begin{equation*}
    I(x) = 1 + xI(x)G(x)
\end{equation*}

\subsection{}
The classes containing the following patterns are Wilf-equivalent
in \(\av{\perm{2,3,1}}\)
\begin{equation*}
    m_1 = \pattern{}{1,2}{0/0,0/1,0/2,
                    1/0,1/1,1/2,
                    2/0,2/1,2/2}
    \text{ and }
    m_2 = \pattern{}{2,1}{0/0,0/1,0/2,
                    1/0,1/1,1/2,
                    2/0,2/1,2/2}
\end{equation*}
It is obvious that these two are Wilf-equivalent since the only
permutations that contain these patterns are \(\perm{1,2}\) and
\(\perm{2,1}\) respectively, therefore the avoiders of these patterns
are counted by the Catalan numbers at all lengths except for length
\(2\) where there is precisely \(1\) avoider.

\subsection{}

The classes containing the following patterns are Wilf-equivalent
in \(\av{\perm{2,3,1}}\)
\begin{equation*}
    m_1 = \pattern{}{2,1}{0/0,0/1,0/2,
                    1/1,1/2,
                    2/0,2/2}
    \text{ and }
    m_2 = \pattern{}{1,2}{0/1,0/2,
                    1/0,1/2,
                    2/0,2/1,2/2}
\end{equation*}

First consider the structure of an avoider of \(m_1\) and \(\perm{2,3,1}\)
we can perform the usual structural decomposition of an avoider
of \(\perm{2,3,1}\) where we consider decomposition around the maximum.
Any permutation in \(\av{\perm{2,3,1},m_1}\) is in the following set
precisely once.
\begin{equation*}
    \mathcal{J}_1 = \varepsilon \sqcup
    \decompmax{0/1,1/0}{\mathcal{C}\setminus\varepsilon}{}{}{\mathcal{C}} \sqcup
    \decompmax{0/1,1/0,0/0}{}{}{}{\mathcal{F}}
\end{equation*}

Now consider the decomposition around the maximum of a permutation in
\(\av{\perm{2,3,1},m_2}\) This fits into the following set.
\begin{equation*}
    \mathcal{J}_2 = \varepsilon \sqcup
    \decompmax{0/1,1/0}{\mathcal{C}}{}{}{\mathcal{C}\setminus\varepsilon} \sqcup
    \decompmax{0/1,1/0,1/1}{\mathcal{F}}{}{}{}
\end{equation*}
Therefore both of these sets of avoiders are enumerated in the same manner
having generating function satisfying
\begin{equation*}
    J(x) = 1 + xC(x)(C(x)-1) + xF(x)
\end{equation*}

\subsection{}
Consider the containers of the patterns
\begin{equation*}
    \pattern{}{1,2}{0/0,0/1,0/2,1/0,1/2,2/0,2/1,2/2},
    \pattern{}{1,2}{1/1,0/1,0/2,1/0,1/2,2/0,2/1,2/2},
    \pattern{}{2,1}{0/0,0/1,0/2,1/0,1/2,1/1,2/1,2/2},
    \pattern{}{2,1}{0/0,0/1,0/2,1/0,1/2,2/0,1/1,2/2},
    \pattern{}{2,1}{0/0,0/1,0/2,1/0,1/2,2/0,2/1,1/1},
    \pattern{}{2,1}{0/0,0/1,0/2,1/1,1/2,2/0,2/1,2/2}
\end{equation*}

For each of these patterns there is precisely one occurrence
in any permutation containing the pattern. Now consider
the points in the free box in each case. Each of these regions
must contain an avoider of \(\perm{2,3,1}\) that is of length
\(n-2\). Therefore these classes are all Wilf-equivalent
and the number of length \(n\) avoiders is
\begin{equation*}
    K_n = C_n - C_{n-2}
\end{equation*}
for \(n \ge 2\) where \(C_n\) is the \(n\)th Catalan number, the number of
\(\perm{2,3,1}\) avoiders of length \(n\).

\subsection{}
Now consider the containers of the patterns
\begin{equation*}
    \pattern{}{2,1}{0/0,0/1,0/2,1/0,1/2,2/0,2/1,2/2},
    \pattern{}{1,2}{0/0,1/1,0/2,1/0,1/2,2/0,2/1,2/2},
    \pattern{}{1,2}{0/0,0/1,1/1,1/0,1/2,2/0,2/1,2/2}
\end{equation*}
Each of these patterns again occurs precisely once in
any containing permutation. However this time when considering
the free box we need to take into consideration \LemmaRef{lem:incdec}
and so the empty box can only contain a decreasing subsequence. There
is precisely one decreasing subsequence at every length, and so there
is exactly one container of each pattern at each length. The three
patterns are Wilf-equivalent and have \(C_n - 1\) avoiders of
length \(n\) for all \(n \ge 2\).

\subsection{}
The containers of the following patterns can only
have exactly one occurrence.
\begin{equation*}
    \pattern{}{1,2}{0/0,0/1,0/2,1/1,2/0,2/1,2/2}
    \pattern{}{2,1}{0/0,0/1,1/0,1/1,1/2,2/0,2/2}
    \pattern{}{1,2}{0/0,0/1,1/1,1/2,2/0,2/1,2/2}
    \pattern{}{1,2}{0/1,0/2,1/0,1/1,2/0,2/1,2/2}
    \pattern{}{2,1}{0/0,0/1,0/2,1/0,1/2,2/0,2/1}
    \pattern{}{1,2}{0/0,0/1,1/0,1/2,2/0,2/1,2/2}
\end{equation*}
Once again we consider the free boxes, obviously for every pattern
except the first the two boxes are independent, and one contains
any avoider of \(\perm{2,3,1}\) and the other must contain a
decreasing sequence by \LemmaRef{lem:incdec}.
Let us consider the first pattern on it's own. In order to
avoid \(\perm{2,3,1}\) across the free boxes we can add some additional
restrictions
\begin{equation}
    \raisebox{0.6ex}{
    \begin{tikzpicture}[scale=1, baseline=(current bounding box.center)]
        \fill[pattern=north east lines, pattern color=black!75] (1.5, 0) rectangle +(0.5,1);
        \fill[pattern=north east lines, pattern color=black!75] (1.0, 2) rectangle +(0.5,1);
        \draw (1.5,0) -- (1.5,1)
              (1.5,2) -- (1.5,3);
        \modpattern[2]{}{1,2}{0/0,0/1,0/2,1/1,2/0,2/1,2/2}
    \end{tikzpicture}
    }
    \label{eqn:splitboxes}
\end{equation}
Now we can see that the top free box must contain a decreasing sequence,
and the bottom must contain an avoider of \(\perm{2,3,1}\) and these
two do not interact in any manner. The containers of this pattern are counted
the same as the other patterns, and due to this they are Wilf-equivalent.

\subsection{}
The classes containing the following patterns are Wilf-equivalent
\begin{equation*}
    \begin{aligned}
        m_1 = \pattern{}{1,2}{0/0,0/1,1/1,1/2,2/0,2/2} &\text{ and }
        m_2 = \pattern{}{1,2}{0/0,0/1,1/0,2/0,2/1,2/2} \\
        \text{ and }
        m_3 = \pattern{}{1,2}{0/2,1/0,1/1,2/0,2/1,2/2} &\text{ and }
        m_4 = \pattern{}{1,2}{0/1,1/0,1/1,2/0,2/1,2/2}
    \end{aligned}
\end{equation*}

First consider the decomposition of avoiders of \(m_1\) in \(\av{\perm{2,3,1}}\)
around the maximum.
\begin{equation*}
    \mathcal{L}_1 =
    \varepsilon \sqcup
    \decompmax{0/1,1/0}{\scriptstyle \mathcal{G}_1\setminus\varepsilon}{}{}{\mathcal{C}}
    \sqcup
    \decompmax{0/0,0/1,1/0}{}{}{}{\mathcal{L}_1}
\end{equation*}

Now we decompose the avoiders of \(m_2\) around the leftmost point
\begin{equation*}
    \mathcal{L}_2 =
    \varepsilon \sqcup
    \decompleft{0/1,1/0}{\mathcal{C}\setminus\varepsilon}{}{}{\mathcal{C}\setminus\varepsilon}
    \sqcup
    \decompleft{0/1,1/0,1/1}{\mathcal{L}_2}{}{}{}
\end{equation*}
This gives us two generating functions satisfying the following pair of equations
\begin{align}
    L_1(x) &= 1 + xC(x)(G(x)-1) +xL_1(x) \label{eqn:genL1}\\
    \text{ and }L_2(x) &= 1 + x(C(x)-1)^2 + xL_2(x) \label{eqn:genL2}
\end{align}
In order for these two functions to give the same value it is necessary to show that
\eqref{eqn:genL1} and \eqref{eqn:genL2} are equal, this occurs if
\(C(x)(G(x)-1) = (C(x)-1)^2\).
\begin{align*}
    &&(C(x)-1)^2 &= C(x)(G(x)-1)\\
        &\Leftrightarrow& x^2C^4(x) &=  xC(x)(C(x)-1)C(x) &&\text{By definition of \(G\) and \(C\)}\\
        &\Leftrightarrow& x^2C^4(x)&=xC^3(x)-xC^2(x)\\
        &\Leftrightarrow& xC^2(x) &= C(x) -1 &&\text{Divide by \(xC^2(x)\)}\\
        &\Leftrightarrow& C(x) &= 1 +xC^2(x)
\end{align*}
The final line is always satisfied since it is the form of \(C(x)\), and
therefore the two generating functions are equal.

Now we look at the other patterns. In particular note that any container of these
patterns can contain the pattern precisely once, \(m_2\) specifies the
minimum and last point, \(m_3\) and \(m_4\) both use the last point and the
previous block bottom (in the set partition context).

Looking at the structure of a container of \(m_3\) in \(\av{\perm{2,3,1}}\)
\begin{equation*}
    \raisebox{0.6ex}{
        \begin{tikzpicture}[scale=1, baseline=(current bounding box.center)]
            \modpattern[2]{}{1,2}{0/2,1/0,1/1,2/0,2/1,2/2}
            \draw (0.5,0.5) node {\(b_1\)};
            \draw (0.5,1.5) node {\(b_2\)};
            \draw (1.5,2.5) node {\(b_3\)};
        \end{tikzpicture}
    }
\end{equation*}
The boxes \(b_2\) and \(b_3\) must contain a decreasing sequence by
\LemmaRef{lem:incdec}. The box labelled \(b_1\) must contain an avoider
of \(\perm{2,3,1}\). However note that the points in this box can have
interaction with any points in box \(b_2\). If there is just one point
in \(b_2\) then any points in \(b_1\) to the left of this point must
be lower than any points to the right of this point. By extension,
if \(b_2\) contains a decreasing sequence with \(k\) points, there
are \(k+1\) non-interacting avoiders of \(\perm{2,3,1}\) in \(b_3\).

Now in \(m_2\) and \(m_4\) containers we can use the same method as in
\eqref{eqn:splitboxes} to separate the two decreasing sequences in the
free boxes in the top row, and the mixing happens in the same manner
as in a container of \(m_3\).
We now have that \(m_2,m_3\) and \(m_4\) have the same number of containers
so are Wilf-equivalent, and that \(m_1\) and \(m_2\) have the same generating
function so all four classes are Wilf-equivalent.
\subsection{}
The classes containing the following patterns are Wilf-equivalent
\begin{equation*}
    \pattern{}{1,2}{} \text{ and }
    \pattern{}{2,1}{}
\end{equation*}
This is true since the only avoiders of these patterns are the decreasing
sequence and the increasing sequence respectively, and both of these
avoid \(\perm{2,3,1}\) in all cases. There is therefore \(1\) avoider at
every length.

\subsection{}
The patterns
\begin{equation*}
    \pattern{}{1,2}{0/0,0/2,1/0,1/2,2/0,2/2},
    \pattern{}{1,2}{1/1,2/1,1/0,1/2,2/0,2/2},
    \pattern{}{1,2}{0/1,1/0,1/2,2/0,2/2},
    \pattern{}{1,2}{0/1,0/2,1/1,1/2,2/0}, \text{ and }
    \pattern{}{1,2}{0/0,0/1,0/2,1/0,1/1,1/2}
\end{equation*}
are Wilf-equivalent.
If \(\mathcal{M}_1\) is the set of avoiders of
\textpattern{}{1,2}{0/0,0/2,1/0,1/2,2/0,2/2}, then by the structural
decomposition around the maximum we have
\begin{equation*}
    \mathcal{M}_1 = \varepsilon \sqcup
    \decompmax{0/0,0/1,1/0}{}{}{}{\mathcal{C}}
\end{equation*}
This is because \(\mathcal{M}_1\) is the set of permutations
who have their minimum occur after their maximum.

% If \(\mathcal{M}_2\) is the set of avoiders of
% \textpattern{}{1,2}{1/1,2/1,1/0,1/2,2/0,2/2}, then by structural
% decomposition around the maximum we have
% \begin{equation*}
%     \mathcal{M}_2 = \varepsilon \sqcup
%     \decompmax{0/1,1/0}{\mathcal{C}}{}{}{\scriptstyle\mathcal{M}_1\setminus\varepsilon}
%     \sqcup \tikz{\filldraw (0,0) circle (3pt);}
% \end{equation*}
% Alternatively,
The pattern \textpattern{}{1,2}{1/1,2/1,1/0,1/2,2/0,2/2} occurs if
the last element is higher than the penultimate element. This can only occur
if the last element is in a single block in the set partition context, In order
to construct a avoider of length \(n\) we can take any avoider of
\(\perm{2,3,1}\) of length \(n-1\) and insert the new maximum into the last
block.  This ensures that the last block is never a singleton. This means
that these permutations are also counted by \(M(x) = 1+xC(x)\).

The avoiders of the third pattern can be decomposed by the maximum
to give
\begin{equation*}
    \mathcal{M}_3 = \varepsilon \sqcup
    \decompmax{0/1,1/0}{\mathcal{F}}{}{}{\mathcal{M}_3}
\end{equation*}
The generating function derived satisfies \(M_3(x) = 1 + xF(x)M_3(x)\).
The fourth pattern can be decomposed around the maximum in a similar
manner.
\begin{equation*}
    \mathcal{M}_4 = \varepsilon \sqcup
    \decompmax{0/1,1/0}{\mathcal{M}_4}{}{}{\mathcal{F}}
\end{equation*}

Finally considering the last pattern, the only way we can construct an
avoider is to take any \(\perm{2,3,1}\) avoider and add a new minimum
at the start of the permutation. Adding a new leftmost point in any other
position would either create an occurrence of \(\perm{2,3,1}\) or the pattern.
Therefore this is also counted by \(M(x) = 1+xC(x)\).

We need to show that the generating function \(M_3(x)\) is the same as \(M(x)\)
\begin{equation*}
    \begin{aligned}
        M_3(x) &= 1 + xF(x)M_3(x) \\
        &= \frac{1}{1-xF(x)} && \text{Solving for \(M_3(x)\)}\\
        &= \frac{1}{1 - \frac{x}{1+x-xC(x)}} &&\text{Substituting for \(F(x)\)}\\
        &= \frac{1-xC(x) +x}{1-xC(x)}\\
        &= 1+xC(x)
    \end{aligned}
\end{equation*}
We have that \(M_3(x) = 1+xC(x)= M(x) \) so all four patterns are Wilf-equivalent.

\subsection{}
The patterns
\begin{equation*}
    \begin{aligned}
        m_1 = \pattern{}{1,2}{0/0,0/1,1/0,1/2,2/1} \text{ and }
        m_2 = \pattern{}{2,1}{0/0,0/1,0/2,1/1,1/2,2/0}\\
        \text{ and }
        m_3 = \pattern{}{1,2}{0/1,0/2,1/0,1/2,2/0,2/2} \text{ and }
        m_4 = \pattern{}{2,1}{1/0,1/1,2/1,2/2}\\
    \end{aligned}
\end{equation*}
Can be shown to be Wilf-equivalent. First we consider a container of \(m_2\) in \(\av{\perm{2,3,1}}\)
\begin{equation*}
    \raisebox{0.6ex}{
    \begin{tikzpicture}[scale=1, baseline=(current bounding box.center)]
        \fill[pattern=north east lines, pattern color=black!75] (2.5, 1) rectangle +(0.5,1);
        \fill[pattern=north east lines, pattern color=black!75] (2.0, 2) rectangle +(0.5,1);
        \draw (2.5,1) -- (2.5,3);
        \draw (1.5,0.5) node {\(b_1\)};
        \draw (2.25,1.5) node {\(\scriptstyle b_2\)};
        \draw (2.75,2.5) node {\(\scriptstyle b_3\)};
        \modpattern[2]{}{2,1}{0/0,0/1,0/2,1/1,1/2,2/0}
    \end{tikzpicture}
    }
\end{equation*}
We can choose the lowest occurrence of \(m_2\) without loss of generality.
The region corresponding to \(b_1\) must avoid the pattern \(\textpattern{}{1}{0/1,1/0}\)
as well as \(\perm{2,1,3}\). The regions corresponding to \(b_2\) and \(b_3\) must
now contain avoiders of \(\perm{2,3,1}\), these regions cannot mix in order to avoid
\(\perm{2,3,1}\). Since we already have an occurrence of
\(m_2\) we do not need to care about creating more occurrences so there are no
other conditions on these boxes.

Now looking at a container of \(m_3\) in \(\av{\perm{2,3,1}}\)
\begin{equation*}
    \raisebox{0.6ex}{
    \begin{tikzpicture}[scale=1, baseline=(current bounding box.center)]
        \fill[pattern=north east lines, pattern color=black!75] (1, 1.5) rectangle +(1,0.5);
        \fill[pattern=north east lines, pattern color=black!75] (2, 1) rectangle +(1,0.5);
        \draw (1,1.5) -- (3,1.5);
        \draw (0.5,0.5) node {\(b_1\)};
        \draw (1.5,1.25) node {\(\scriptstyle b_2\)};
        \draw (2.5,1.75) node {\(\scriptstyle b_3\)};
        \modpattern[2]{}{1,2}{0/1,0/2,1/0,1/2,2/0,2/2}
    \end{tikzpicture}
    }
\end{equation*}
We consider the leftmost occurrence of \(m_3\), the region corresponding
to \(b_1\) must avoid the pattern \(\textpattern{}{1}{0/1,1/0}\) as well as
\(\perm{2,3,1}\) once more. The regions corresponding to \(b_2\) and \(b_3\)
must avoid \(\perm{2,3,1}\) and as in a container of \(m_2\) these regions cannot
mix, as doing so would lead to an occurrence of \(\perm{2,3,1}\).
Therefore both of these sets of containers are enumerated in the same way.

Now we find a structural decomposition for an avoider of \(m_2\).
Decomposing around the maximum we see the set of avoiders of \(m_2\)
have the form
\begin{equation*}
    \mathcal{N}_2 = \varepsilon \sqcup
    \decompmax{0/1,1/0}{\scriptstyle\mathcal{N}_2\setminus\varepsilon}{}{}{\mathcal{C}}
    \sqcup
    \decompmax{0/0,0/1,1/0}{}{}{}{\mathcal{F}}
\end{equation*}
We can decompose an avoider of \(m_1\) in \(\av{\perm{2,3,1}}\) around the leftmost point
in a similar manner
\begin{equation*}
    \mathcal{N}_1 = \varepsilon \sqcup
    \decompleft{0/1,1/0}{\scriptstyle\mathcal{N}_1\setminus\varepsilon}{}{}{\mathcal{C}}
    \sqcup
    \decompleft{0/0,0/1,1/0}{}{}{}{\mathcal{F}}
\end{equation*}
These two decompositions tell us that these two patterns are Wilf-equivalent
and have generating function
\begin{equation}
    N_1 = 1 + x(N_1(x)-1)C(x) +xF(x)
    \label{eqn:ngen}
\end{equation}
Now consider an avoider of \(m_4\) decomposed around the maximum
\begin{equation*}
    \mathcal{N}_4 = \varepsilon \sqcup
    \decompmax{0/1,1/0}{\mathcal{C}}{}{}{\mathcal{O}}
\end{equation*}
Where \(\mathcal{O}\) is the permutations avoiding \(\perm{2,3,1},m_4\) and
\(p = \textpattern{}{1}{0/0,0/1,1/1}\) since if the subsequence in this box
were to start with the maximum then this point and the maximum would create
an occurrence of \(m_4\).
Now consider decomposition of a permutation in \(\mathcal{O}\) around its
leftmost point.
\begin{equation*}
    \mathcal{O} = \varepsilon \sqcup
    \decompleft{0/1,1/0}{\mathcal{C}}{}{}{\scriptstyle\mathcal{N}_4\setminus\varepsilon}
\end{equation*}
This gives us the generating function
\begin{equation*}
    O(x) = 1 + xC(x)(N_4(x)-1)
\end{equation*}
Now we can construct the following for \(N_4\)
\begin{equation}
    N_4(x) = 1 +xC(x)(xC(x)(N_4(x)-1)+1)
    \label{eqn:ngen2}
\end{equation}

All that remains to show Wilf-equivalence is to show that \EquationRef{eqn:ngen} and
\EquationRef{eqn:ngen2} are the same generating function.
First solve \EquationRef{eqn:ngen2} for \(N_4(x)\)
\begin{equation}
    \begin{aligned}
        N_4(x) &= 1+xC(x)(xC(x)(N_4(x)-1)+1)\\
        &= 1 + x^2C^2(x)N_4(x) -x^2C^2(x) +xC(x)\\
        &= 1 + \frac{xC(x)}{1 - x^2C^2(x)}\\
        &= 1 + \frac{xC(x)}{(1-xC(x))(1+xC(x))} && \text{Difference of squares}\\
        N_4(x)&= 1 + \frac{xC^2(x)}{1+xC(x)} && C(x) = \frac{1}{1-xC(x)}
    \end{aligned}
\end{equation}

Now we solve \EquationRef{eqn:ngen} for \(N_1(x)\)
\begin{equation}
    \begin{aligned}
        N_1(x)&= 1+x(N_1(x)-1)C(x)+xF(x)\\
        &= 1+xN_1(x)C(x)-xC(x)+\frac{x}{1+x-xC(x)} &&\text{Substitution of \(F(x)\)}\\
        N_1(x)(1-xC(x))&= \frac{x^2C^2(x) - (x^2+2x)C(x) + 2x+1}{1+x-xC(x)}\\
        N_1(x) &= \frac{x^2C^2(x) - (x^2+2x)C(x) + x+1+x}{x^2C^2(x) - (x^2+2x)C(x) +x+1}\\
        &= 1 +\frac{x}{x^2C^2(x) - (x^2+2x)C(x) +x+1}\\
        &= 1 + \frac{x}{1-x^2C(x)-xC(x)} && xC^2(x) = C(x)-1\\
        &= 1 + \frac{xC^2(x)}{C^2(x) - xC^3(x)(x+1)}\\
        &= 1 + \frac{xC^2(x)}{C(x) - xC^2(x) + xC(x)} && xC^2(x) = C(x)-1\\
        N_1(x)&= 1 + \frac{xC^2(x)}{1+xC(x)} && C(x) = 1+xC^2(x)\\
    \end{aligned}
\end{equation}
We have shown that \(N_1\) and \(N_4\) are indeed the same generating function,
and we have that the classes containing these four patterns are Wilf-equivalent.

\subsection{}
\begin{equation*}
    m_1 = \pattern{}{1,2}{1/0,1/1,1/2,2/0,2/1} \text{ and }
    m_2 = \pattern{}{2,1}{0/1,1/0,1/1,1/2,2/1}
\end{equation*}
Let \(\mathcal{P}_1\) be the set of avoiders of \(m_1\), by structural decomposition
around the leftmost point we have
\begin{equation*}
    \mathcal{P}_1 = \varepsilon \sqcup
    \decompleft{0/1,1/0}{\mathcal{P}_1}{}{}{\mathcal{P}^\prime_1}
\end{equation*}
Where \(\mathcal{P}^\prime_1\) is a permutation avoiding \(\perm{2,3,1}, m_1\)
and \(\textpattern{}{1}{0/0,0/1,1/1}\)
Now consider the decomposition of a permutation in \(\mathcal{P}^\prime_1\) it can
once again be decomposed around the leftmost point
\begin{equation*}
    \mathcal{P}^\prime_1 = \varepsilon \sqcup
    \decompleft{0/1,1/0}{\scriptstyle \mathcal{P}_1\setminus\varepsilon}{}{}{\mathcal{P}^\prime_1}
\end{equation*}
This is a complete decomposition of avoiders of \(m_1\). Pow we
look at an avoider of \(m_2\), this time decomposition is around the maximum
\begin{equation*}
    \mathcal{P}_2 = \varepsilon \sqcup
    \decompleft{0/1,1/0}{\mathcal{P}_2}{}{}{\mathcal{P}^\prime_2}
\end{equation*}

Again we use the same method of decomposition of a permutation in \(\mathcal{P}^\prime_2\)
\begin{equation*}
    \mathcal{P}^\prime_2 = \varepsilon \sqcup
    \decompleft{0/1,1/0}{\scriptstyle \mathcal{P}_2\setminus\varepsilon}{}{}{\mathcal{P}^\prime_2}
\end{equation*}

This gives us a generating function \(P(x)\) satisfying
\begin{align}
    P(x) &= 1 + xP(x)P^\prime(x) \label{eqn:Pgen}\\
    P(x) &= 1 + x(P(x)-1)P^\prime(x)\label{eqn:PprimeGen}
\end{align}

Solving \EquationRef{eqn:PprimeGen} for \(P^\prime(x)\) and substituting into
\EquationRef{eqn:Pgen} gives us the fact that a the generating function for
\(P(x)\) satisfies
\begin{equation}
    P(x) = xP^2(x) - x(P(x) - 1) + 1
\end{equation}

\subsection{}
\begin{gather*}
        \pattern{}{2,1}{0/0,0/1,0/2,1/0,1/1,1/2,2/2},
        \pattern{}{1,2}{0/1,0/2,1/0,1/1,1/2,2/0,2/2},
        \pattern{}{2,1}{0/0,0/1,0/2,1/0,1/1,1/2,2/1},
        \pattern{}{2,1}{0/0,0/1,0/2,1/1,1/2,2/0,2/1},
        \pattern{}{2,1}{0/0,0/1,0/2,1/0,1/1,1/2,2/0},\\
        \pattern{}{1,2}{0/0,0/1,0/2,1/2,2/0,2/1,2/2},
        \pattern{}{1,2}{0/0,0/1,0/2,1/1,1/2,2/0,2/2},
        \pattern{}{1,2}{0/0,0/1,1/0,1/1,1/2,2/0,2/1},
        \pattern{}{1,2}{0/0,0/1,1/0,1/1,1/2,2/0,2/1}
\end{gather*}

In order to gain enumeration, consider decomposition of avoiders
of \(\textpattern{}{2,1}{0/0,0/1,0/2,1/0,1/1,1/2,2/2}\) around
the maximum.
\begin{equation*}
    \mathcal{Q}_1 = \varepsilon \sqcup
    \tikz{\filldraw (0,0) circle (3pt);}
    \sqcup
    \decompmax{0/1,1/0}{\mathcal{C\setminus\varepsilon}}{}{}{\mathcal{C}}
\end{equation*}

This gives us the following generating function
\begin{equation}
    Q(x) = 1 + x + xC(x)(C(x)-1) \label{eq:qgen}
\end{equation}

Now we consider decomposition of an avoider
\textpattern{}{1,2}{0/1,0/2,1/0,1/1,1/2,2/0,2/2} around the maximum,
this avoider must fit into the following form
\begin{equation*}
    \mathcal{Q}_2 = \varepsilon \sqcup
    \decompmax{0/1,1/0}{\mathcal{G}_2}{}{}{\mathcal{C}}
\end{equation*}
This gives us the generating function
\begin{equation*}
    \begin{aligned}
        Q_2(x) &= 1+xC(x)G(x)\\
        &= 1+xC(x)(1+xC(x)(C(x)-1))\\
        &= 1 +xC(x)(C(x)-xC(x)) && C(x) = 1+xC^2(x)\\
        &= 1 +x + xC^2(x) - xC(x) \\
        &= 1 +x + xC(x)(C(x)-1)
    \end{aligned}
\end{equation*}
Therefore this generating function is the same as \EquationRef{eq:qgen}.
We can decompose
\textpattern{}{2,1}{0/0,0/1,0/2,1/0,1/1,1/2,2/1},
\textpattern{}{2,1}{0/0,0/1,0/2,1/1,1/2,2/0,2/1},
\textpattern{}{2,1}{0/0,0/1,0/2,1/0,1/1,1/2,2/0},
\textpattern{}{1,2}{0/0,0/1,0/2,1/2,2/0,2/1,2/2},
and \textpattern{}{1,2}{0/0,0/1,0/2,1/1,1/2,2/0,2/2} around the leftmost point
into an avoider of one of the patterns with generating function \(G(x)\) and
an avoider of \(\perm{2,3,1}\).

Now decompose an avoider of \textpattern{}{1,2}{0/0,0/1,1/0,1/1,1/2,2/0,2/1}
around the leftmost point.
\begin{equation*}
    \mathcal{Q}_3 = \varepsilon \sqcup
    \decompleft{0/1,1/0}{\scriptstyle \mathcal{Q}\setminus\varepsilon}{}{}{\mathcal{C}}
    \sqcup \decompleft{0/0,0/1,1/0}{}{}{}{\mathcal{G}_3}
\end{equation*}
This gives generating function \(Q_3(x)\) satisfying
\begin{equation*}
    \begin{aligned}
        Q(x) &= 1 +xC(x)(P(x)-1) + xG(x)\\
        &= C(x) - xC^2(x) + xC(x) +x^2C^3(x) - x^2C^2 && \text{Solving for \(Q(x)\)}\\
        &= 1 + xC(x) +x^2C^3(x)-xC^2(x) && C = 1+xC^2(x)\\
        &= 1 + x + x^2C^3(x) && xC^2(x) = C(x)-1\\
        &= 1 + x + xC(x)(C(x)-1)
    \end{aligned}
\end{equation*}
This is equivalent to \EquationRef{eq:qgen}, and therefore
the classes containing all of these patterns are Wilf-equivalent.
