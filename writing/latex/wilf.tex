\chapter{Wilf-equivalences under dominating classical patterns}

Another question often asked in thefield of permutation patterns is that
of Wilf-equivalence. Two patterns \(\pi\) and \(\sigma\) are said to be
Wilf-equivalent if their avoidance sets have the same size at each
length. More formally

\begin{definition}[Wilf-equivalence]
    Two patterns \(\pi\) and \(\sigma\) are said to be \emph{Wilf-equivalent}
    if for all \(k_{} \ge 0\), \(\size{\av[k]{\pi}} = \size{\av[k]{\sigma}}\).

    Two sets of permutation patterns \(R\) and \(S\) are are
    \emph{Wilf-equivalent} if for all \(k_{} \ge 0\),
    \(\size{\av[k]{R}} = \size{\av[k]{S}}\).
\end{definition}

Wilf-equivalence is of interest as if two permutation classes are enumerated
in the same way then there should exist a bijection between them, and therefore
any other combinatorial object that they represent.

There are a number of symmetries we can use when examining Wilf-equivalences
to reduce the amount of work, it can be easily seen that the reverse, complement
and inverse operations preserve enumeration, and therefore these classes are trivially
Wilf-equivalent.
Since we are always considering Wilf-equivalences in the set \(\av{\pi}\) we
must only use these symmetries when they preserve the dominating pattern.
