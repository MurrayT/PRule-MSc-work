\chapter{Conclusions and Future work}
From \ChapterRef{chap:coincs} it is can be seen that automatically classifying
coincidences of mesh patterns is a difficult task, establishing rules for longer
dominating patterns requires many more cases to be taken. It would however
be interesting to consider the self application of the Dominating Rules to mesh
patterns in order to try to capture some of the coincidences described in
\textcite{DBLP:journals/combinatorics/HilmarssonJSVU15} and \textcite{journals/combinatorics/BrandenC11}.

It is also possible to take sets of mesh patterns instead of a single mesh pattern
when considering dominating rules, and expressing coincidence between these sets.
Doing this may give nice enumerative results.

It would be interesting to consider a systematic explanation of Wilf-equivalences
amongst classes where \(\perm{3,2,1}\) is the dominating pattern using the
construction presented in \cite[Sec.~12]{2015arXiv151203226B}, in order to directly
reach enumeration and hopefully establish some of the non-trivial Wilf-equivalences
between classes with different dominating patterns.
